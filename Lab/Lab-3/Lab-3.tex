% template created by: Russell Haering. arr. Joseph Crop
\documentclass[12pt,letterpaper]{article}
\usepackage{anysize}
\marginsize{2cm}{2cm}{1cm}{1cm}

\begin{document}

\begin{titlepage}
    \vspace*{4cm}
    \begin{flushright}
    {\huge
        ECE 375 Lab 3\\[1cm]
    }
    {\large
       – Introduction to AVR Simulation with Atmel Studio
    }
    \end{flushright}
    \begin{flushleft}
    Lab Time: Thursday  10:00 - 11:50
    \end{flushleft}
    \begin{flushright}
    Jeremy Fischer

    \vfill
    \rule{5in}{.5mm}\\
    TA Signature
    \end{flushright}

\end{titlepage}

\section{Additional Questions}
\begin{enumerate}
	\item 
\textbf{		What is the initial value of DDRB? }
		
		DDRB is at 0x37 and has an initial value of 0x00
		
	
	
	\item 
\textbf{	What is the initial value of PORTB? }
	
	PORTB is at 0x38 and has an initial value of 0x00
	
	
	
	\item
\textbf{	Based on the initial values of DDRB and PORTB, what is Port B’s default I/O configuration? }
	
	Port B's default I/0 configuration is all 0's, or, set to output
	
	
	\item 
\textbf{	What 16-bit address was the stack pointer just initialized to? }
	
	The stack pointer was initialized to 0x10FF
	
	\item 
\textbf{	What are the current contents of register r0? }
	
	r0 is initialized to 0xFF
	
	
	
	\item 
\textbf{	How many times did the code inside the loop structure end up running? }
	
	The loop was ran five times
	
	
	\item 
\textbf{	Which instruction would you modify if you wanted to change the number of times that the loop runs? }
	
	To change the number of times the loop is ran I would change the ldi i, \$04 instruction. Specifically the counter, \$04 


	\item 
\textbf{	What are the current contents of register r1? }
	
	After the loop, the contents of r1 is 0xAA
	
	
	\item 
\textbf{	What are the current contents of register r2? }
	
	After loop2, the contents of r2 is 0x0F
	
	
	\item 
\textbf{	What are the current contents of register r3? }
	
	The current contents of r3 is 0x0F 
	
	
	
	\item 
\textbf{	What is the value of the stack pointer now that your program flow has moved inside of a subroutine?}
	
	The stack pointer's value is 0x10FD
	
	
	
	\item 
\textbf{	What is the final result of FUNCTION? (What are the hexadecimal contents of memory locations \$0105:\$0104?)}
	
	 0x0105 contains 0e and 0x0104 contains aa


\end{enumerate}

\end{document}
