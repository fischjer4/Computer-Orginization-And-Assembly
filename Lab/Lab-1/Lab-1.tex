% template created by: Russell Haering. arr. Joseph Crop
\documentclass[12pt,letterpaper]{article}
\usepackage{anysize}
\marginsize{2cm}{2cm}{1cm}{1cm}

\begin{document}

\begin{titlepage}
    \vspace*{4cm}
    \begin{flushright}
    {\huge
        ECE 375 Lab 1\\[1cm]
    }
    {\large
        Introduction to AVR Tools
    }
    \end{flushright}
    \begin{flushleft}
    Lab Time: Thursday 10:00 - 11:50
    \end{flushleft}
    \begin{flushright}
    Jeremy Fischer
    
    \vfill
    \rule{5in}{.5mm}\\
    TA Signature
    \end{flushright}

\end{titlepage}


\section{Study Questions}
\begin{enumerate}
    \item
    Go to the lab webpage and download the template write-up. 
    Read it thoroughly and get familiar with the expected format. 
    \textbf{What specific font is used for source code, and at what size?} 
    \textit{From here on, when you include your source code in your lab write-up, you must adhere to the specified font type and size.}

    The font that is used is Courier New, and the font size is eight
    
    \item
    Go to the lab webpage and read Syllabus carefully. 
    Expected format and naming convention are very important for submission. 
    If you do not follow naming conventions and formats, you will lose some points. 
    \textbf{What is the naming convention for source code (asm)? 
    What is the naming convention for source code files if you are working with your partner?}

	The naming convention for source code is \textit{First\_name\_Last name\_Lab\#\_sourcecode.asm}
	
	The naming convention for source code when working with a partner is \textit{First\_name\_Last name\_and\_First\_name\_Last name\_Lab\#\_sourcecode.asm}

	\item 
	Take a look at the code you downloaded for today’s lab. 
	Notice the lines that begin with .def and .equ followed by some type of expression. 
	These are known as \textbf{pre-compiler directives}. 
	Define pre-compiler directive. \textbf{What is the difference between the .def and .equ directives?} 
	(HINT: see Section 5.1 of the AVR Starter Guide).

	A pre-compiler directive is a special instruction that is executed before the code is compiled.
	These special instructions direct the compiler to adjust the location of the program in memory or define macros that will be used in the soon to be compiled code.
	
	The difference between .equ and .def is that .equ sets a symbol to an expression where .def sets a symbolic name on a register.
	
	
	\item 
	Take another look at the code you downloaded for today’s lab. 
	Read the comment that describes the macro definitions. 
	From that explanation, determine the 8-bit binary value that each of the following expressions evaluates
	to. Note: the numbers below are decimal values
	\begin{enumerate}
		\item 
		(1 $<<$ 3)
		
		00001000
		
		\item 
		(2 $<<$2)
		
		00001000
		
		\item 
		(8 $>>$ 1)
		
		00000100
		
		\item 
		(1 $<<$ 0)
		
		00000001
		
		\item 
		(6 $>>$ 1$|$1 $<<$ 6)

		01000011
		
	\end{enumerate}
	
	
	
\end{enumerate}


\end{document}
