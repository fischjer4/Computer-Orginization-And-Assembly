% template created by: Russell Haering. arr. Joseph Crop
\documentclass[12pt,letterpaper]{article}
\usepackage{anysize}
\usepackage{graphicx}
\usepackage{enumerate}
\marginsize{2cm}{2cm}{1cm}{1cm}


\usepackage{listings}

\begin{document}

\begin{titlepage}
    \vspace*{4cm}
    \begin{flushright}
    {\huge
        ECE 375 Pre-Lab 5\\[1cm]
    }
    {\large
     	Large Number Arithmetic
    }
    \end{flushright}
    \begin{flushleft}
    Lab Time: Thursday 10:00 - 11:50
    \end{flushleft}
    \begin{flushright}
    Jeremy Fischer
    
    \vfill
    \rule{5in}{.5mm}\\
    TA Signature
    \end{flushright}

\end{titlepage}


\section{Pre-Lab}

	\begin{enumerate}
		\item 
		For this lab, you will be asked to perform arithmetic operations on numbers that are \textbf{larger than 8 bits.} 
		To be successful at this, you will need to understand and utilize many of the various arithmetic operations supported by the AVR 8-bit instruction set. 
		List and describe \textbf{all} of the addition, subtraction, and multiplication instructions (i.e. ADC, SUBI, FMUL, etc.) available in AVR’s 8-bit instruction set.
		
		\textit{All descriptions of the below instructions were taken from microchip.com \cite{micro}}
		
		\begin{itemize}
			\item 
			\textbf{ADC:} Adds two registers and the contents of the C flag and places the result in the destination register Rd.
			
			\item 
			\textbf{ADD:} Adds two registers without the C flag and places the result in the destination register Rd.
			
			\item 
			\textbf{ADIW:} Adds an immediate value (0-63) to a register pair and places the result in the register pair. This instruction operates on the upper four register pairs, and is well suited for operations on the pointer registers.
			
			\item 
			\textbf{FMUL:} This instruction performs 8-bit x 8-bit $\rightarrow$ 16-bit unsigned multiplication and shifts the result one bit left.
			
			\item 
			\textbf{FMULS:} This instruction performs 8-bit x 8-bit $\rightarrow$ 16-bit signed multiplication and shifts the result one bit left.
			
			\item 
			\textbf{FMULSU:} This instruction performs 8-bit x 8-bit $\rightarrow$ 16-bit signed multiplication and shifts the result one bit left.
			
			\item 
			\textbf{MUL:} This instruction performs 8-bit x 8-bit $\rightarrow$ 16-bit unsigned multiplication.
			
			\item 
			\textbf{MULS:} This instruction performs 8-bit x 8-bit $\rightarrow$ 16-bit signed multiplication.
			
			\item 
			\textbf{MULSU:} This instruction performs 8-bit x 8-bit $\rightarrow$ 16-bit multiplication of a signed and an unsigned number.
			
			\item 
			\textbf{SBC:} Subtracts two registers and subtracts with the C flag and places the result in the destination register Rd.
			
			\item 
			\textbf{SBCI:} Subtracts a constant from a register and subtracts with the C flag and places the result in the destination register Rd.
			
			\item 
			\textbf{SBIW:} Subtracts an immediate value (0-63) from a register pair and places the result in the register pair. This instruction operates on the upper four register pairs, and is well suited for operations on the pointer registers.
			
			\item 
			\textbf{SUB:} Subtracts two registers and places the result in the destination register Rd.
			
			\item 
			\textbf{SUBI:} 
			Subtracts a register and a constant and places the result in the destination register Rd. This instruction is working on Register R16 to R31 and is very well suited for operations on the X, Y and Z pointers.
			
		\end{itemize}
	
		\clearpage
		
		\item 
		Write pseudocode for an 8-bit AVR function that will take two 16-bit numbers (from data memory addresses \$0111:\$0110 and \$0121:\$0120),\textit{ add
		them together}, and then store the 16-bit result (in data memory addresses \$0101:\$0100). 
		(Note: The syntax “\$0111:\$0110” is meant to specify that the function will expect \textit{little-endian} data, where the highest byte of a multi-byte value is stored in the highest address of its range of addresses.)

		\begin{lstlisting}
			Func:
				;Set R4 to all 0's
				clr R4
				
				;Get values from memory
				R0 = M[$0110] ;R1:R0
				R1 = M[$0111]
				R2 = M[$0120] ;R3:R2d
				R3 = M[$0121]
				
				;Add R1:R0 to R3:R2
				add R2, R0 ; Add low byte
				adc R3, R1 ; Add with carry high byte
				
				;Store result
				M[$0100] = R3	;result
				M[$0101] = R4	;possible carry
		\end{lstlisting}
	
	
		\item 
		Write pseudocode for an 8-bit AVR function that will take the 16-bit number in \$0111:\$0110,\textbf{ subtract it from} the 16-bit number in \$0121:\$0120, and then store the 16-bit result into \$0101:\$0100.
	
		\begin{lstlisting}
			Func:
				;Get values from memory
				R0 = M[$0110] ;R1:R0
				R1 = M[$0111]
				R2 = M[$0120] ;R3:R2d
				R3 = M[$0121]
			
				; Subtract R1:R0 from R3:R2
				sub R2, R0 ; Subtract low byte
				sbc R3, R1 ; Subtract with carry high byte
			
				;Store result
				M[$0100] = R3	;result
		\end{lstlisting}
	
	
	
	\end{enumerate}

	\bibliographystyle{IEEEtran}
	\bibliography{references.bib}

\end{document}

