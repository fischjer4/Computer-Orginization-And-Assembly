% template created by: Russell Haering. arr. Joseph Crop
\documentclass[12pt,letterpaper]{article}
\usepackage{anysize}
\usepackage{graphicx}
\usepackage{enumerate}
\marginsize{2cm}{2cm}{1cm}{1cm}


\usepackage{listings}
\usepackage{amsmath}
\usepackage{nicefrac}


\begin{document}

\begin{titlepage}
    \vspace*{4cm}
    \begin{flushright}
    {\huge
        ECE 375 Pre-Lab 8\\[1cm]
    }
    {\large
     	 Remotely Operated Vehicle (USART)
    }
    \end{flushright}
    \begin{flushleft}
    Lab Time: Thursday 10:00 - 11:50
    \end{flushleft}
    \begin{flushright}
    Jeremy Fischer
    
    \vfill
    \rule{5in}{.5mm}\\
    TA Signature
    \end{flushright}

\end{titlepage}


\section{Pre-Lab}

	\begin{enumerate}
		\item
		\textit{In this lab, you will be given a set of behaviors/actions that you need to have a proof-of-concept “toy” perform. 
		Think of a toy you know of (or look around online for a toy) that is likely implemented using a microcontroller, and \textbf{describe the behaviors it performs}. 
		Here is an example behavior: “If you press button X on the toy, it takes action Y (or makes sound Z)”.}
		
		Any baby toy that plays music, or makes a sound, based off of the button that was pressed.
		Think of a toy in the shape of a barn.
		When the pig shaped button is pressed the barn speakers output "oink oink."
		When the cow shaped button is pressed the barn speakers output "mooooooo."
		And when the dog shaped button is pressed the barn speakers output "roof roof."
		
		
		\item
		\textit{For each behavior you described in the previous question, explain which microcontroller feature was likely used to implement that behavior, and give a brief code example indicating how that feature should be configured. 
		Make your explanation as \textbf{ATmega128-specific as possible} (e.g., discuss which I/O registers would need to be configured, and if any interrupts will be used), and also mention if any additional mechanical and/or electronic devices are needed.}
		
		This is most likely set up the exact same way our AtMega128 button presses and LEDs are set up.
		The buttons could be set up as input on Port D with the pull-up mode set.
		And the two speakers could be setup as output on Port B.
		Each of the button presses could be set up via interrupts, so when the pig button is pressed the microcontroller sends the recorded pig sound to the speakers. 
		
		\begin{verbatim}
			Setup Pig interrupt
			Setup Cow interrupt
			Setup Dog interrupt
			
			Set up bits 0, 1, and 2 on Port D for input
			Set up bits 0, 1, and 2 on  Port D as pull-up
			
			Set up bits 0 and 1 on Port B as output
			
			MAIN:
				rjmp MAIN
				
			PIG_ISR:
				Play Pig Sound
				Unstage staged interrupts
			
			COW_ISR:
				Play Cow Sound
				Unstage staged interrupts
			
			DOG_ISR:
				Play Dog Sound
				Unstage staged interrupts
				
		\end{verbatim}
		
		
		\item
		\textit{Each ATmega128 USART module has two flags used to indicate its current \textbf{transmitter} state: the Data Register Empty (UDRE) flag and Transmit Complete (TXC) flag. 
		What is the difference between these two flags, and which one always gets set first as the transmitter runs? 
		You will probably need to read about the\textit{ Data Transmission} process in the datasheet (including looking at any relevant USART diagrams) to answer this question.}
		
		UDRE is set when the byte that is in UDR moves from UDR to the transmit shift register. 
		It is important to note that this doesn't mean that the byte has been transmitted yet.
		All it means is that UDR is now clear for the next byte to be written. 
		TXC on the other hand is set when the byte has actual been transmitted.
		To answer the question about which gets set first, it is UDRE since it is set when the data moves to the transmit shift register.
		
		
		
		
		\item
		\textit{Each ATmega128 USART module has one flag used to indicate its current \textbf{receiver} state (not including the error flags). \textbf{For USART1 specifically}, what is the name of this flag, and what is the interrupt vector address for the interrupt associated with this flag? 
		This time, you will probably need to read about Data Reception in the datasheet to answer this question.}
		
		RXC1 (USART Receive Complete) is the flag which indicates the current reciever state.
		The vector number is 31 and the vector address is 0x003C.
		
		
		
	\end{enumerate}
\end{document}

