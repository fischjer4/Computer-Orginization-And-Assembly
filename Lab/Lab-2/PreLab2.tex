% template created by: Russell Haering. arr. Joseph Crop
\documentclass[12pt,letterpaper]{article}
\usepackage{anysize}
\marginsize{2cm}{2cm}{1cm}{1cm}

\begin{document}

\begin{titlepage}
    \vspace*{4cm}
    \begin{flushright}
    {\huge
        ECE 375 Pre-Lab 2\\[1cm]
    }
    {\large
       Pre-Lab: C$->$Assembly$->$Machine Code$->$TekBot
    }
    \end{flushright}
    \begin{flushleft}
    Lab Time: Thursday 10:00 - 11:50
    \end{flushleft}
    \begin{flushright}
    Jeremy Fischer
    
    \vfill
    \rule{5in}{.5mm}\\
    TA Signature
    \end{flushright}

\end{titlepage}


\section{Pre-Lab}

\begin{enumerate}
	\item 
	Suppose you want to configure Port B so that all 8 of its pins are configured as outputs. 
	Which I/O register is used to make this configuration, and what 8-bit binary value must be written to configure all 8 pins as outputs?
	
	DDRB is the register used to configure Port B's pins to outputs.
	
	The value that must be written to DDRB to configure all 8 pins as outputs is b11111111, or decimal 255.
	
	\item 
	Suppose all 8 of Port D's pins have been configured as inputs. 
	Which I/O register must be used to read the current state of Port D's pins?
	
	PIND is the I/O register that must be used to read the current state of Port D's pins.
	
	\item 
	 Does the function of a PORTx register differ depending on the setting of its corresponding DDRx register? 
	 If so, explain any differences
	 
	 Yes. If DDRx configures port X to input, then writing a 1 to PORTxn (where n is the specified I/0 pin) activates a pull-up resistor.
\end{enumerate}


\end{document}
