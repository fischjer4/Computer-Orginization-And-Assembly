% template created by: Russell Haering. arr. Joseph Crop
\documentclass[12pt,letterpaper]{article}
\usepackage{anysize}
\usepackage{graphicx}
\usepackage{enumerate}
\marginsize{2cm}{2cm}{1cm}{1cm}

\begin{document}

\begin{titlepage}
    \vspace*{4cm}
    \begin{flushright}
    {\huge
        ECE 375 Pre-Lab 4\\[1cm]
    }
    {\large
     	Data Manipulation \& the LCD
    }
    \end{flushright}
    \begin{flushleft}
    Lab Time: Thursday 10:00 - 11:50
    \end{flushleft}
    \begin{flushright}
    Jeremy Fischer
    
    \vfill
    \rule{5in}{.5mm}\\
    TA Signature
    \end{flushright}

\end{titlepage}


\section{Pre-Lab}

\begin{enumerate}
	\item 
	What is the stack pointer? 
	How is the stack pointer used, and how do you initialize it? 
	Provide pseudocode (\textbf{not} actual assembly code) that illustrates how to initialize the stack pointer.
	
	The stack pointer points to the top of the stack frame and is responsible for keeping track of the return address and local variables.
	
	\begin{enumerate}
		\item Load the low byte of of the end of the SRAM Address to R\textsubscript{r}.
		\item Write the R\textsubscript{r} to the Stack Pointer Low byte.
		\item Load the high byte of of the end of the SRAM Address to R\textsubscript{r}.
		\item Write the R\textsubscript{r} to the Stack Pointer high byte.
	\end{enumerate}
	
	
	\item 
	What does the AVR instruction LPM do, and how do you use it?
	Provide pseudocode (\textbf{not} actual assembly code) that shows how to setup and use the LPM instruction.

	LPM stands for Load Program Memory and this instruction loads the (one) byte pointed to by the Z register into a destination register.
	
	\begin{enumerate}
		\item Load the address which you want to read from into the Z register.
		\item Store the contents of the destination register in a TEMP register to retrieve later.
		\item LPM destination register, Z register.
	\end{enumerate}
	
	
	\item 
	Take a look at the definition file m128def.inc (This file can be found in the \textbf{Solution Explorer $ \rightarrow$  Dependencies} folder in Atmel Studio, assuming you have an Assembler project open and you have already built an assembly program that includes this definition file. 
	Two good examples of such a project would be your Lab 1 and Lab 3 projects.) What is contained within this definition file? 
	What are some of the benefits of using a definition file like this? 
	Please be specific, and give a couple examples if possible.

	m128def.inc contains definitions for registers and I/0 ports, so users can refer to them by their name instead of their physical address. From the document description itself:
	
	\begin{itemize}
	 	\item \textit{When including this file in the assembly program file, all I/O register names and I/O register bit names appearing in the data book can be used.}
		\item\textit{ In addition, the six registers forming the three data pointers X, Y and Z have been assigned names XL - ZH. Highest RAM address for Internal }
		\item\textit{ SRAM is also defined}
	\end{itemize}

	The benefits of using a definition file like this include using registers and I/O ports by name. Specifically, instead of saying 0x16 one could simply say PINB.
	Or, instead of saying r31 one could say ZH.
	Using a definition file makes reading and understanding the code easier.


\end{enumerate}


\end{document}
