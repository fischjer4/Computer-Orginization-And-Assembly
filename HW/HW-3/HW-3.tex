\documentclass[12pt,letterpaper]{article}
\usepackage{anysize}
\usepackage{graphicx}
\usepackage{enumerate}
\marginsize{2cm}{2cm}{1cm}{1cm}

\usepackage{color}
\usepackage{listings}
\usepackage[utf8]{inputenc}
\usepackage{longtable}

\begin{document}

\begin{titlepage}
    \vspace*{4cm}
    \begin{flushright}
    {\huge
        ECE 375 Homework 3\\[1cm]
    }
    {\large
       Computer Organization and Assembly Language Programming
    }
    \end{flushright}
    \begin{flushleft}
    Winter Term 2018
    \end{flushleft}
    \begin{flushright}
    Jeremy Fischer

    \end{flushright}

\end{titlepage}


\section{Homework Questions}
	\begin{enumerate}
	    %Question 1
	    \item
		The AVR code below (with some information missing) is designed to initialize and service interrupts from three I/O devices (DevA, DevB, and DevC). 
	
		\begin{enumerate}[a]
			\item
			There are 8 external interrupt pins (INT0-INT7) in AVR. 
			Which of the three interrupt pins are these I/O devices connected to and what is the immediate value needed in line (1)?
			
			\textbf{Answer:}
			The three I/O devices are connnected to INT0, INT3, and INT5.
			The immediate value needed in line (1) is: 0b00101001.
			
			
			\item
			Which I/O device’s interrupt is detected on a falling edge?
			
			\textbf{Answer:}
			INT3 is detected on a falling edge.
			
			
			\item
			The interrupt pins referred to in part (a) are connected to two of the 7 ports (PORTA-PORTG) in AVR.
			Which ports are they?
			
			\textbf{Answer:}
			INT0 and INT3 are connected to PORTD and INT5 is connected to PORTE.
			
			
			\item
			There are important instructions missing in lines 2-3 of the code. 
			Fill in the missing instructions in lines 2-3 so that the code will work correctly.
			
			\textbf{Answer:} Shown Below.
			
			\item 
			Suppose DevA requires that no interrupts are detected while it is being serviced. 
			Fill in lines 4-5 with the necessary code to clear any latched interrupts at the end of ISR\_DevA.
			
			\textbf{Answer:} Shown Below.
			
			
			\item
			Suppose interrupts are detected from all three interrupt pins at the same time. 
			Which subroutine (ISR\_DevA, ISR\_DevB, or ISR\_DevC) will be executed first? 
			
			\textbf{Answer:} INT0 would be triggered, meaning that ISR\_DevA would be called.
			
		\end{enumerate}

	   \lstinputlisting{Q1.asm}
	   
	  
	
		\clearpage
		%Question 2
	   \item
	 	Consider the WAIT subroutine for Tekbot discussed in class that waits for 1 sec. and returns. 
	 	Rewrite the WAIT subroutine so that it waits for 1 sec. using the 16-bit Timer/Counter1 with the highest resolution.
	 	Assume that the system clock frequency is 16 MHz and the Timer/Counter1 is operating under the CTC
	 	mode. 
	 	This is done by doing the following: 
	 	
	 	(a) Timer/Counter1 is initialized to operate in the CTC mode.
	 	
	 	(b) The WAIT subroutine loads the proper value into OCR1A and waits until OCF1 is set. Once OCF1 is set, it is cleared and the WAIT subroutine returns.
	 	
	 	Use the skeleton code shown below. Also, show the necessary calculations for determining \textit{value} and \textit{prescale}. 
	 	Note that your code may not use any other GPRs besides mpr.
	 	
	 	
		\textbf{Answer:}
		\lstinputlisting{Q2.asm}
		
		
		\clearpage
		%Question 3
		\item
		Consider the following AVR assembly code that waits for 1 sec. using the 8-bit Timer/Counter0 with the system clock frequency of 16 MHz operating under Normal mode. Write and explain the instructions in lines (1-10) necessary to make this code work properly. 
		More specifically, 
		\begin{enumerate}[a]
			\item 
			Fill in lines (1-2) with the necessary code to enable the interrupt on Timer/Counter0 overflow.
			
			\item
			Fill in lines (3-4) with the necessary code to load the value for delay that generates an interrupt after 5
			ms.
			
			\item 
			Fill in lines (5-6) with the necessary code to set the prescalar value and the mode of operation.
			
			\item 
			Fill in line (7) with the necessary code to set counter.
			
			\item
			Fill in lines (8-9) with the necessary code to reload the value for delay.
			
			\item
			Fill in line (10) with the necessary code to decrement counter.
		\end{enumerate}
		
		\textbf{Answer:}
		\lstinputlisting{Q3.asm}
		
		\clearpage
		
		\item
		Consider the AVR code segment shown below that performs initialization and receive routines for USART1 to continuously receives data from a transmitter. Write and explain the instructions in lines (1-16) necessary to make this code work properly. 
		More specifically,
		\begin{enumerate}[a]
			\item 
			Fill in lines (1-4) with the necessary code to initialize the stack pointer.
			
			\item 
			Fill in lines (5-6) with the necessary code to configure RXD1 (Port D, pin 2).
			
			\item 
			Fill in lines (7-10) with the necessary code to set the Baud rate at 9,600 bps using double data rate.
			
			\item 
			Fill in lines (11-12) with the necessary code to enable the receiver and the Receive Complete interrupt.
			
			\item 
			Fill in lines (13-14) to set the frame format to be asynchronous with 8 data bits, 2 stop bits, and even parity.
			
			\item 
			Fill in lines (15-16) to receive data and store it into Buffer.
		\end{enumerate}
		Assume the system clock frequency of 16 MHz.
		
		\textbf{Answer:}
		\lstinputlisting{Q4.asm}
		
	\end{enumerate}

\end{document}
