% template created by: Russell Haering. arr. Joseph Crop
\documentclass[12pt,letterpaper]{article}
\usepackage{anysize}
\usepackage{graphicx}
\usepackage{enumerate}
\marginsize{2cm}{2cm}{1cm}{1cm}

\usepackage{color}
\usepackage{listings}
\usepackage[utf8]{inputenc}
\usepackage{longtable}
\usepackage{multirow}
\begin{document}

\begin{titlepage}
    \vspace*{4cm}
    \begin{flushright}
    {\huge
        ECE 375 Homework 4\\[1cm]
    }
    {\large
       Computer Organization and Assembly Language Programming
    }
    \end{flushright}
    \begin{flushleft}
    Winter Term 2018
    \end{flushleft}
    \begin{flushright}
    Jeremy Fischer

    \end{flushright}

\end{titlepage}


\section{Homework Questions}
	The following questions are based on the enhanced AVR datapath (see Figures 8.24 and 8.26 in the text). 
	The microoperation for the Fetch cycle is shown below.
	
	\begin{tabular}{ |c|c| } 
			\hline
			Stage & Micro-operations \\ \hline
			IF & IR $\leftarrow$ M[PC], PC $\leftarrow$ PC + 1, NPC $\leftarrow$ PC + 1, RAR $\leftarrow$ PC + 1  \\ \hline
	\end{tabular}

	\begin{enumerate}
	    %Question 1
	    \item
	    Consider the implementation of the \textit{CLR Rd} (Clear Register) instruction on the enhanced AVR datapath.
	    \begin{enumerate}[a]
	    	\item
	    	List and explain the sequence of microoperations required to implement \textit{CLR Rd}. Note that this
	    	instruction takes a single execute cycle (EX).
	    	
	    	\item
	    	List and explain the control signals and the Register Address Logic (RAL) output for the \textit{CLR Rd} instruction.
	    	Control signals for the Fetch cycle are given below. 
	    	Clearly explain your reasoning.
	    \end{enumerate}

		\textbf{Answer:}
		
		Rd $\leftarrow$ Rd $\oplus$ Rd
		\\
		Simply EOR the source register, Rd, with itself.
		
		\begin{itemize}
			\item MA = 1 due to Rd duplicate coming from the $||$	 operator.
			\item ALU\_f = 1010 for EOR
			\item RF\_wB = 1 due to writing to wB (Rd)
			\item DM\_w = 0 to disallow writing to Data Memory
		\end{itemize}
   		\begin{tabular}{ |c|c|c| } 
	   		\hline
	   		\multicolumn{3}{|c|}{CLR}\\ \hline
	   		Control Signals & IF & EX \\ \hline
	   		MJ & 0  & x \\ \hline
	   		MK & 0 &  x \\ \hline
	   		ML & 0 &  x \\ \hline
	   		IR\_en& 1 & x \\ \hline
	   		PC\_en & 1 & 0  \\ \hline
	   		PCh\_en & 0 & 0  \\ \hline
	   		PCI\_en  & 0 &  0 \\ \hline
	   		NPC\_en  & 1 &  x \\ \hline
	   		SP\_en  & 0 &  0 \\ \hline
	   		DEMUX  & x &  x \\ \hline
	   		MA  & x &  1 \\ \hline
	   		MB  & x &  0 \\ \hline
	   		ALU\_f  & xxxx & 1010\\ \hline
	   		MC  & xx &  00 \\ \hline
	   		RF\_wA  & 0 &  0 \\ \hline
	   		RF\_wB  & 0 &  1 \\ \hline
	   		MD  & x &  x \\ \hline
	   		ME  & x &  x \\ \hline
	   		DM\_r  & x & x \\ \hline
	   		DM\_w  & 0 &  0 \\ \hline
	   		MF  & x &  x \\ \hline
	   		MG  & x &  x \\ \hline
	   		Adder\_f  & xx &  xx\\ \hline
	   		Inc\_Dec  & x &  x \\ \hline
	   		MH  & x &  x \\ \hline
	   		MI  & x &  x \\ \hline
 		\end{tabular}
	 	\quad
	 	\quad
		\begin{tabular}{ |c|c|} 
			\hline
			\multicolumn{2}{|c|}{CLR}\\ \hline
			RAL output & EX \\ \hline
			wA & x  \\ \hline
			wB &  Rd\\ \hline
			rA &  x \\ \hline
			rB &  Rd\\ \hline
		\end{tabular}
	  
	
		\clearpage
		%Question 2
	   \item
		Consider the implementation of the \textit{STD Y+q, Rr} (Store Indirect with Displacement) instruction on the enhanced AVR datapath.
		
		
		\begin{enumerate}[a]
			\item
			List and explain the sequence of microoperations required to implement \textit{STD Y+q, Rr}. 
			Note that this instruction takes two execute cycles (EX1 and EX2).
			
			\item
			List and explain the control signals and the Register Address Logic (RAL) output for the \textit{STD Y+q, Rr} instruction.
		\end{enumerate}
		Control signals for the Fetch cycle are given below. Clearly explain your reasoning.
		
		\textbf{Answer:} 
		
			\textit{First cycle:} Get the address to store Rr, (Y+q)
			\\
			\textit{Second cycle:} then store Rr there.

			DMAR $\leftarrow$ AR + q
			\\
			M[DMAR] $\leftarrow$ Rr 
				
		
		\begin{itemize}
			\item In the first execution cycle we only need to get YH:YL and then add it to q inside of the Address Adder and output the result to DMAR.
			\item In the second cycle we only need to read Rr, and store it in M[DMAR].
			\item You can see the above take place in the RAL table and EX1/EX2 columns in the control signals table.
		\end{itemize}
	
		\begin{tabular}{ |c|c|c|c| } 
			\hline
			\multicolumn{4}{|c|}{STD Y+q, Rr}\\ \hline
			Control Signals & IF & EX1 & EX2 \\ \hline
			MJ & 0  &  x & x \\ \hline
			MK & 0 &  x & x \\ \hline
			ML & 0 &  x & x \\ \hline
			IR\_en& 1 & 0 & x \\ \hline
			PC\_en & 1 & 0 & 0\\ \hline
			PCh\_en & 0 & 0 & 0 \\ \hline
			PCI\_en  & 0 & 0 & 0 \\ \hline
			NPC\_en  & 1 & x & x \\ \hline
			SP\_en  & 0 & 0 & 0 \\ \hline
			DEMUX  & x & x & x \\ \hline
			MA  & x & x &  x \\ \hline
			MB  & x & x & x \\ \hline
			ALU\_f  & xxxx & xxxx &  xxxx\\ \hline
			MC  & xx & xx & xx \\ \hline
			RF\_wA  & 0 & 0 & 0 \\ \hline
			RF\_wB  & 0 & 0 & 0 \\ \hline
			MD  & x & x & 1 \\ \hline
			ME  & x & x & 1 \\ \hline
			DM\_r  & x & x & 0 \\ \hline
			DM\_w  & 0 & 0 & 1 \\ \hline
			MF  & x & 1 &  x \\ \hline
			MG  & x & 1 & x \\ \hline
			Adder\_f  & xx & 00 & xx \\ \hline
			Inc\_Dec  & x & x &  x \\ \hline
			MH  & x & 1 &  x \\ \hline
			MI  & x & x & x\\ \hline
		\end{tabular}
		\quad
		\quad
		\begin{tabular}{ |c|c|c|} 
			\hline
			\multicolumn{3}{|c|}{STD Y+q, Rr}\\ \hline
			RAL output & EX1 & EX2 \\ \hline
			wA & x &  x \\ \hline
			wB & x &  x \\ \hline
			rA &  YH & Rr \\ \hline
			rB &  YL & x \\ \hline
		\end{tabular}
		
		
		
		
		
		
		\clearpage
		%Question 3
		\item
		Consider the implementation of the \textit{RCALL k} (\textit{Relative Call to Subroutine}) instruction on the enhanced AVR datapath.
		
		\begin{enumerate}[a]
			\item
			List and explain the sequence of microoperations required to implement \textit{RCALL k}. 
			Note that this instruction takes two execute cycles (EX1 and EX2).
			
			\item
			List and explain the control signals and the Register Address Logic (RAL) output for the \textit{RCALL k} instruction
		\end{enumerate}
		Control signals for the Fetch cycle are given below. Clearly explain your reasoning.
		
		\textbf{Answer:} 
		
		\textit{First cycle:} RAR gets PC + 1, PC points to (PC + k)
		\\
		\textit{Second cycle:} Jump to (PC + k)
		
		NPC $\leftarrow$ M[PC], RAR $\leftarrow$ PC + 1, M[SP] $\leftarrow$ RARL, SP $\leftarrow$ SP - 1
		\\
		PC $\leftarrow$ NPC, M[SP] $\leftarrow$ RARL, SP $\leftarrow$ SP - 1
		
		\begin{tabular}{ |c|c|c|c| } 
			\hline
			\multicolumn{4}{|c|}{RCALL}\\ \hline
			Control Signals & IF & EX1 & EX2 \\ \hline
			MJ & 0  & 0 & 1 \\ \hline
			MK & 0 &  0  & x \\ \hline
			ML & 0 & 0 & 0 \\ \hline
			IR\_en& 1 & 1 &  x \\ \hline
			PC\_en & 1 &  1 & x \\ \hline
			PCh\_en & 0 & 0 &  0 \\ \hline
			PCI\_en  & 0 & 0 &  0 \\ \hline											
			NPC\_en  & 1 & 1 & 0 \\ \hline
			SP\_en  & 0 & 1 & 1 \\ \hline
			DEMUX  & x & x & x \\ \hline
			MA  & x & x &  x \\ \hline
			MB  & x & x  & x \\ \hline
			ALU\_f  & xxxx & xxxx & xxxx\\ \hline
			MC  & xx & xx & xx\\ \hline
			RF\_wA  & 0 & 0 & 0 \\ \hline
			RF\_wB  & 0 & 0 & 0 \\ \hline
			MD  & x &  0 & 0 \\ \hline
			ME  & x &  0 & 0 \\ \hline
			DM\_r  & x & x & x \\ \hline
			DM\_w  & 0 & 1 & 1 \\ \hline
			MF  & x & 0 & x \\ \hline
			MG  & x &  0 & x \\ \hline
			Adder\_f  & xx & 00 & xx \\ \hline
			Inc\_Dec  & x & 1 & 1 \\ \hline
			MH  & x & x &  x \\ \hline
			MI  & x & 0 & 1 \\ \hline
		\end{tabular}
		\quad
		\quad
		\begin{tabular}{ |c|c|c|} 
			\hline
			\multicolumn{3}{|c|}{RCALL}\\ \hline
			RAL output & EX1 & EX2 \\ \hline
			wA & x &  x \\ \hline
			wB & x &  x\\ \hline
			rA & x & x \\ \hline
			rB & x & x \\ \hline
		\end{tabular}








		\clearpage
		%Question 4
		\item
			Consider the implementation of the \textit{LPM R16, Z+} (Load Program Memory) instruction on the enhanced AVR datapath. 
		
		\begin{enumerate}[a]
			\item
			List and explain the sequence of microoperations required to implement \textit{LPM R16, Z+}. 
			Note that this instruction takes three execute cycles (EX1, EX2, and EX3). 
			
			\item
			List and explain the control signals and the Register Address Logic (RAL) output for the LPM instruction.
			Control signals for the Fetch cycle are given below. Clearly explain your reasoning. 
		\end{enumerate}
		Control signals for the Fetch cycle are given below. Clearly explain your reasoning.
		
		\textbf{Answer:} 
		
		PMAR $\leftarrow$ Z, PC $\leftarrow$ PC + 1
		\\
		MDR $\leftarrow$ PM[PMAR], ZH:ZL $\leftarrow$ (ZH:ZL) + 1
		\\
		R16 $\leftarrow$ MDR
		
		
		\textit{First cycle:}  Read the address in ZH:ZL., PC = PC + 1
		\\
		After the first cycle PMAR will have ZH:ZL
		
		\textit{Second cycle:} MDR $\leftarrow$ PM[ZH:ZL], Z $\leftarrow$ Z + 1
		\\
		After the second cycle MDR will have PM[Z], and Z will have been incremented and stored.
		
		\textit{Third cycle:} R16 $\leftarrow$ MDR 
		\\
		After the third cycle R16 will have the value of MDR
		
		\begin{itemize}
			\item First cycle MUXL must be done to read ZH:ZL from PMAR
			\item Second cycle MUXC must be 01 to inB will be reading ZL from the address adder
			\item Third cycle MUXC must be 10 to so inB will be MDR
		\end{itemize}

		\begin{tabular}{ |c|c|c|c|c| } 
			\hline
			\multicolumn{5}{|c|}{LPM}\\ \hline
			Control Signals & IF & EX1 & EX2 & EX3\\ \hline
			MJ & 0  & x & x &  x \\ \hline
			MK & 0 &  x & x &  x\\ \hline
			ML & 0 &  x & 1 &  x \\ \hline
			IR\_en& 1 & 0 & 0  &  x \\ \hline
			PC\_en & 1 & 0 & 0 &  0 \\ \hline
			PCh\_en & 0 & 0 & 0  & 0 \\ \hline
			PCI\_en  & 0 & 0 & 0 & 0 \\ \hline
			NPC\_en  & 1 & x & x &  x \\ \hline
			SP\_en  & 0 & 0 & 0 &  0 \\ \hline
			DEMUX  & x &  x & x & x \\ \hline
			MA  & x & x & x &  x \\ \hline
			MB  & x & x  & x &  x \\ \hline
			ALU\_f  & xxxx & xxxx & xxxx &  xxxx \\ \hline
			MC  & xx &  xx & 01 &  10 \\ \hline
			RF\_wA  & 0 & 0 & 1 &  0 \\ \hline
			RF\_wB  & 0 &  0 & 1 &  1 \\ \hline
			MD  & x & x & x &  x \\ \hline
			ME  & x & x & x & x \\ \hline
			DM\_r  & x & x & x & x \\ \hline
			DM\_w  & 0 & 0 & 0 &  0 \\ \hline
			MF  & x & x & x &  x \\ \hline
			MG  & x & 1 & 1 &  x \\ \hline
			Adder\_f  & xx & 11  & 01 &  xx \\ \hline
			Inc\_Dec  & x & x & x &   x\\ \hline
			MH  & x & 1 & x &  x \\ \hline
			MI  & x & x & x &   x\\ \hline
		\end{tabular}
		\quad
		\quad
		\begin{tabular}{ |c|c|c|c|} 
			\hline
			\multicolumn{4}{|c|}{LPM}\\ \hline
			RAL output & EX1 & EX2 & EX3\\ \hline
			wA & x &  ZH &  x \\ \hline
			wB & x  & ZL &  R16 \\ \hline
			rA & ZH  & ZH & x \\ \hline
			rB & ZL   & ZL &  x\\ \hline
		\end{tabular}	
		





		
	\end{enumerate}

\end{document}
